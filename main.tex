\documentclass[
    a4paper,
    12pt,
    headinclude=true,
    BCOR=10mm,
    %toc=bibnumbered
]{scrreprt}

\usepackage[utf8]{inputenc}
\usepackage[T1]{fontenc}
\usepackage{microtype}
\usepackage{listings}
\lstset{
    basicstyle=\ttfamily,
    literate={Ö}{{\"O}}1 {Ä}{{\"A}}1 {Ü}{{\"U}}1 {ß}{{\ss}}1 {ü}{{\"u}}1 {ä}{{\"a}}1 {ö}{{\"o}}1
}
\usepackage{longtable}
\usepackage{booktabs}
\usepackage{xcolor}
\usepackage[german]{babel}
\usepackage{setspace}
\usepackage[hidelinks]{hyperref}
\usepackage{csquotes}
\usepackage{svg}
\usepackage[backend=biber,style=authoryear,language=german]{biblatex}
\addbibresource{refs/refs.bib}
\ExecuteBibliographyOptions{date=year}
\renewbibmacro*{doi+eprint+url}{%
  \iftoggle{bbx:doi}
    {\printfield{doi}}
    {}%
  \newunit\newblock
  \iftoggle{bbx:eprint}
    {\usebibmacro{eprint}}
    {}%
  \newunit\newblock
  \iftoggle{bbx:url}
    {\iffieldundef{doi}
      {\usebibmacro{url+urldate}}
      {}}
    {}}
\renewbibmacro*{url+urldate}{%
  \printfield{url}%
  \ifentrytype{online}
    {\setunit*{\addspace}%
     \usebibmacro{urldate}}
    {\ifentrytype{misc}
      {\iffieldundef{doi}
        {\setunit*{\addspace}%
         \usebibmacro{urldate}}
        {}}
      {}}}
\DefineBibliographyStrings{german}{
    andothers = {et\addabbrvspace al\adddot},
    and = {und},
}

\usepackage[acronym]{glossaries}
\makeglossaries
\newacronym{nft}{NFT}{Non-Fungible Token}

\usepackage{fancyhdr}
\pagestyle{fancy}
\fancyhf{}
\rhead{\today}
\lhead{\namen}
\cfoot{\thepage}

\onehalfspacing

% User-defined commands
\newcommand{\namen}{H. Ehrmann}

\begin{document}
\begin{titlepage}
    \centering
    \vspace*{1cm}
    \Large{\textbf{PLATZHALTER FÜR TITEL}}\\
    \vspace{1.5cm}
    \Large{Fachhochschule Potsdam}\\
    \vspace{0.5cm}
    \large{Informations- und Datenmanagement}\\
    \vspace{1.5cm}
    Holger Ehrmann\\
    Matr. Nr. 21766\\
    \vspace{1cm}
    D11 Informationsintegration\\
    Sommersemester 25\\
    Prof. Dr. Günther Neher
    \vfill
    \today
\end{titlepage}
\setcounter{tocdepth}{1}
\begingroup
\setstretch{1.2} % Setzt den Zeilenabstand im Inhaltsverzeichnis
{\small \tableofcontents}
\endgroup

\chapter{Einleitung}

Die Umsetzung des Datenintegrationsprozesses erfolgt in einer isolierten Python-Umgebung, die mit den Bibliotheken \texttt{pandas}, \texttt{numpy}, \texttt{openpyxl}, \texttt{fuzzywuzzy}, \texttt{sqlalchemy}, \texttt{mysqlclient}, \texttt{psycopg2-binary} und \texttt{nltk} ausgestattet ist.

Die Rohdaten liegen als CSV-Datei mit dem Feldtrenner \texttt{\_§\_} vor und werden mit \texttt{pandas.read\_csv} unter expliziter UTF-8-Kodierung eingelesen. Im Anschluss werden führende sowie abschließende Unterstriche aus den Spaltennamen entfernt und numerische Datentypen, etwa \texttt{jobid} sowie die Koordinaten \texttt{geo\_lat} und \texttt{geo\_lon}, korrekt konvertiert; das Datumsfeld \texttt{date} wird in ein standardisiertes Zeitformat überführt. Diese Normalisierung schafft eine konsistente Basis für alle weiteren Verarbeitungsschritte.

Zunächst wird ein umfangreiches Data Profiling durchgeführt, das strukturelle und inhaltliche Anomalien der Ausgangsdaten sichtbar macht. Auf dieser Grundlage folgen schrittweise Bereinigungen und Normalisierungen, wobei \texttt{fuzzywuzzy} zur Erkennung und Zusammenführung von Dubletten eingesetzt wird. Für die Anreicherung der Daten mit regionalen Informationen werden externe Referenzdaten eingebunden und über \texttt{sqlalchemy} mit den bestehenden Datensätzen verknüpft. Abschließend ermöglichen Skripte zur Extraktion von Freitextinformationen die Überführung unstrukturierter Angaben in auswertbare Felder. Diese methodische Kette gewährleistet eine transparente und reproduzierbare Aufbereitung des Datensatzes.

\newpage
\addcontentsline{toc}{chapter}{Eigenständigkeitserklärung}
\chapter*{Eigenständigkeitserklärung}
Ich erkläre, dass
\begin{itemize}
    \item ich die schriftliche Prüfungsleistung (Hausarbeit und sonstige schriftliche Ausarbeitungen im Rahmen eines Leistungsnachweises) oder den von mir verantworteten und namentlich kenntlich gemachten Teil im Rahmen einer Gruppenarbeit selbstständig verfasst habe,
    \item ich keine anderen als die angegebenen Hilfsmittel und Quellen benutzt habe,
    \item Teile der Arbeit oder die Arbeit an sich nicht an anderer Stelle als Prüfungsleistung vorgelegt wurde und
    \item die Passagen der Arbeit, die fremden Werken wörtlich oder sinngemäß entnommen sind, unter Angabe der Quellen und unter Beachtung der im Wissenschaftsbereich geltenden allgemeinen verwendeten Zitierregelungen gekennzeichnet sind.
\end{itemize}

Zugleich erkläre ich, dass ich die Prüfungsleistung vor der Abgabe der Leistung bei QIS angemeldet habe.

\vspace{2cm}

\noindent
Potsdam, den \today

\vspace{2cm}

\noindent
\rule{6cm}{0.4pt}\\
Unterschrift

\addcontentsline{toc}{chapter}{Literaturverzeichnis}
\printbibliography
\newpage
\appendix
\end{document}
